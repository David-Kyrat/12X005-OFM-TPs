\documentclass[a4paper, titlepage]{article}
\usepackage[round, sort, numbers]{natbib}
\usepackage[utf8]{inputenc}
\usepackage{amsfonts, amsmath, amssymb, amsthm}
\usepackage{color}
\usepackage{listings}
\usepackage{mathtools}
\usepackage{multicol}
\usepackage{paralist}
\usepackage{parskip}
\usepackage{subfig}
\usepackage{tikz}
\usepackage{titlesec}

\numberwithin{figure}{section}
\numberwithin{table}{section}

\usetikzlibrary{arrows, automata, backgrounds, petri, positioning}
\tikzstyle{place}=[circle, draw=blue!50, fill=blue!20, thick]
\tikzstyle{marking}=[circle, draw=blue!50, thick, align=center]
\tikzstyle{transition}=[rectangle, draw=black!50, fill=black!20, thick]

% define new commands for sets and tuple
\newcommand{\setof}[1]{\ensuremath{\left \{ #1 \right \}}}
\newcommand{\tuple}[1]{\ensuremath{\left \langle #1 \right \rangle }}
\newcommand{\card}[1]{\ensuremath{\left \vert #1 \right \vert }}

\lstdefinestyle{mystyle}{
    basicstyle=\ttfamily\footnotesize,
    breakatwhitespace=false,         
    breaklines=true,                 
    captionpos=b,                    
    keepspaces=true,                 
    showspaces=false,                
    showstringspaces=false,
    showtabs=false,                  
    tabsize=2
}

\lstset{style=mystyle}

\definecolor{lstbg}{rgb}{1,1,0.9}
\lstset{basicstyle=\ttfamily, numberstyle=\tiny, breaklines=true, backgroundcolor=\color{lstbg}, frame=single}
\lstset{language=C}

\makeatletter
\newcommand\objective[1]{\def\@objective{#1}}
\newcommand{\makecustomtitle}{%
	\begin{center}
		\huge\@title \\
		[1ex]\small Aurélien Coet, Dimitri Racordon \\
	\end{center}
	\@objective
}
\makeatother

\begin{document}

  \title{Outils formels de Modélisation \\ 12\textsuperscript{ème} séance d'exercices}
  \author{Arélien Coet, Dimitri Racordon}
	\objective{
    Dans cette séance d'exercices, nous allons étudier la logique du premier ordre.
    Nous tâcherons notamment de traduire des phrases d'un domaine externe à la logique
    en formules du premier ordre, et inversément.
  }

	\makecustomtitle

  \section{logic.translate.com?fr-fol ($\bigstar$)}
    Traduisez les phrases suivantes en formules de logique du premier ordre.
    Créez autant de prédicats et/ou de formules que nécessaire.
    \begin{enumerate}
      \item Il y a un assistant qui n'est pas une femme.
      \item Tous les étudiants sont soit des hommes, soit des femmes.
      \item Tous les étudiants qui étudient les méthodes formelles sont intelligents.
      \item Aucun étudiant n'est meilleur que tous les autres étudiants.
    \end{enumerate}

  \section{logic.translate.com?fol-fr ($\bigstar\bigstar$)}
    Traduisez les formules suivantes en phrases.
    \begin{enumerate}
      \item $\forall a, (Homme(a) \implies Barbe(a)) \land (Femme(a) \implies \lnot Barbe(a))$
      \item $\exists a, \exists b, \exists c, Soeur(a,b) \land Soeur(b,c) \land Soeur(c,a)$
      \item $\forall x, \forall y, BelleSoeur(x,y) \implies \exists z, Femme(x) \land Epouse(y,z) \land Soeur(x,z)$
      \item $\forall x, \forall y, Enfant(x) \land Pokemon(y) \implies Aime(x,y)$
    \end{enumerate}

  \pagebreak

  \section{Un peu de F\# ($\bigstar\bigstar\bigstar$)}
    La logique du premier ordre permet d'exprimer des formules sur des ensembles de tailles arbitraires,
    éventuellement inifinis,
    par l'utilisation des quantificateurs existentiel ($\exists$) et universel ($\forall$).
    En F\#, on peut traduire cela par l'utilisation de prédicats sur des séquences:

    \begin{lstlisting}
    let isWoman (name: string) : bool =
        match name with
        | "Aline" | "Cynthia" -> true
        | _ -> false

    let people =
        [ "Aline"; "Bernard"; "Cynthia" ]

    printfn 
        $"There are women in the list of people:
          {List.exists isWoman people}"

    printfn
        $"There are only women in the list of people: 
          {List.forall isWoman people}"
    \end{lstlisting}

    En partant de ce principe, écrivez les formules de logique des exercices 1 et 2 en F\#.

\end{document}
